\documentclass{book}
\usepackage[T1]{fontenc}
\usepackage[Black]{Alegreya}

\begin{document}

\section{Introduction}

In this game you play a stalwart adventurer seeking riches in the dread Dungeon of Chaos. To win the game, you must be the first to collect 2000 gold coins and return to town. The game is designed for 2 to 4 players.

\section{Components}

To play Chaos Dungeons you require:

\begin{itemize}
\item Four 6x8 Dungeon Boards.
\item The four Monsters Decks and four Item Decks.
\item The Magic Spell, Chest, Scroll, and Potion Decks.
\item The Room Deck, the Event Deck, and the Door Deck.
\item The Dungeon Tile Deck, and the 2x2 and 1x2 Dungeon Tile Decks.
\item The Character Profile Deck, the Curse Deck, and the Gods Deck.
\item The 2x2 Entry Tile and the special 1x1 tiles.
\item The Vampire, Zombie, and Werewolf Special Rules cards.
\item Tokens to represent the players and some dungeon features, such as chests and secret doors.
\item Some eight-sided dice (known hereafter as d8’s) and six-sided dice (known as d6’s).
\item A character sheet and pencil for each player.
\item Plastic stands to place cards on the dungeon board that are too large for the squares.  
\end{itemize}

\section{The Dungeon}

The Dungeon of Chaos is a complex of rooms, corridors, and tunnels filled with monsters, treasure, and traps. The top level is dungeon level 1, and the bottom level is dungeon level 4. Each of the 6x8 Dungeon Boards represents one of the four levels. At the beginning of the game, the boards are empty except for the 2x2 Entry Tile. As players explore the dungeon, they place tiles from the Dungeon Tile Deck on the board. The short sides of each board are numbered 1 through 6, and the long sides 1 through 8. If you have to determine a random square on a floor (as the result of teleportation, for example), roll d6 and d8 and locate the square where the corresponding row and column meet. When a player follows stairs down, falls through a pit, or otherwise moves to a lower floor, he moves his token to the same square on the board representing the next level. Moving up from dungeon level 1 leads outside (see §\ref{town}. The Town). If a room has an exit leading off one side of a board, you can move through that exit and come out on the square on the same board directly opposite.

The traps and enemies become more dangerous as one delves lower into the dungeon. The enemies on level 1 are easiest to defeat, but if you spend too much time on the first level you will fall behind other players who are exploring level 2. The monsters on level 3 are very dangerous and you should be well equipped before venturing there. It is not recommended that players explore level 4.  

\section{The Decks}

The cards in The Dungeon of Chaos serve several purposes. The Scroll, Potion, and Item cards represent possessions the players can discover in the dungeon. The Dungeon Tile Cards are placed on the boards as players explore the dungeon. When a player opens a chest, he draws a Chest Card and follows its instructions, then discards the card.  There are separate Monster and Item decks for each level, which contain the monsters and useful items found on those levels. The usage of each of the decks is described in detail below. If a deck runs out of cards, shuffle the discards to form a new deck. Sometimes cards need to be placed on the board to represent dungeon features or room contents. Most of the cards are too large to fit on a tile; use the plastic stand to place cards on the board.

\section{Beginning the Game}
Each player draws three Character Profile Cards and chooses one. Shuffle the cards that are not chosen back into the Character Profile deck. Then each player whose character has the ability \emph{spellcasting} draws a number of Spell Cards equal to his character’s Mind. Place the entry tile on the centre four squares of one of the boards. All players choose a token to represent their character and place it on the entry tile. Finally, each player rolls d8 and adds his character’s Speed. The player with the highest total goes first; roll again in the case of a tie.  

\section{The Character Profile}

The cards in the Character Profile deck contain descriptions of the characters whose roles you play. These include your five attributes: Attack, Defence, Speed, Mind, and Health, as well as a list of character abilities.
Some abilities are written in /textbf{copperplate gothic} followed by a colon. These abilities are combat actions (see §\ref{combat}. Combat). If your character can cast spells he will have the ability \emph{spellcasting} (see §\ref{spellcasting}.  Spells).
A character with the ability \emph{flying} can avoid pits, chasms, and other hazards - see the descriptions of the Dungeon Tiles for more information.

\section{Attributes}

Monsters and characters both have Attack, Defence, Speed, Mind, and Health. Effects that refer to your “starting values” refer to the values printed on your Character Profile. These attributes will change over the course of your adventure, and can go above their starting values.
Your Mind represents both your mental acuity and willpower. It governs the casting of spells and your resistance to a number of enemy abilities. If your Mind is reduced to 0 or less, you lose your sanity and perish in the dungeon.
Your Speed governs how far your move on your turn, your initiative rolls, and your chance of fleeing successfully from enemies. If your Speed is reduced to 0 or less, you become paralyzed and are quickly devoured by the dungeon’s denizens. (Some creatures begin with 0 Mind or 0 Speed. These creatures ignore effects that reduce these attributes.)
Your Attack and Defence are chiefly relevant in battle (see §\ref{combat}. Combat). They can both be less than 0.
Your Health represents how much damage your character can sustain before dying. It decreases as you are wounded in the dungeon by monsters and traps. If your Health reaches 0 or less, your character dies, and you must quit or start again with another character (see §\ref{dying}. Dying). Effects that “restore” your Health cannot bring this attribute above its starting value. However, you can “gain” Health and end up with more than your starting value.
Some rules refer to the “effective” value of one of your attributes. For example, while \emph{asleep} or \emph{surprised}, your effective Defence is 0. Effective values are used to compute the results of game actions, such as attacking or rolling for initiative. Effective values are temporary and do not change the actual value of an attribute. Effects that change your attributes permanently change the actual value of your attribute, not the effective value.
An example will make this rule clearer. Suppose Barry is playing a Wizard, who has 7 Mind. Barry drinks an unidentified potion, which turns out to be a Potion of Intoxication, and he becomes \emph{drunk}. While drunk, Barry has an effective Mind of 1, and will have trouble getting much use out of his spells. Barry is then attacked by a Mind Drinker, which reduces his Mind by 1. Barry’s effective Mind does not go to 0, causing him to lose his sanity. This reduction instead comes off Barry actual value of 7, and his effective continues to be equal to 1 as long as he is drunk. When Barry later sobers up, his Mind value will be equal to 6.

\section{Turns}
Your turn is divided into three phases: the \emph{Beginning Phase}, the \emph{Movement Phase}, and the \emph{Combat Phase}. You start your turn in the Beginning Phase, and then pass into the Movement Phase. The Combat Phase begins when you encounter a monster.

In the Beginning Phase, you first perform forced effects in any order you choose. Then you can use any ability you have that involves skipping your turn. If you do, play passes to the next player. Otherwise you enter the movement phase of your turn.

You begin the movement phase with a number of \emph{movement actions} equal to two plus your Speed. Moving your token between dungeons tiles expends a movement action. Other movement actions, such as searching for secret doors, are described below and on cards.
Between movement actions, any number of \emph{non-movement actions} can be performed. Most actions are non-movement actions; an action is a non-movement action unless it is specifically referred to as a movement action.  For example, drinking a potion and reading a scroll are both non-movement actions. Other non-movement actions are described below and on cards.
While you can perform any number of non-movement actions during the movement phase, they cannot be performed during the resolution of other actions. For example, you step into a room containing a pit. You cannot stop to drink a Potion of Flight before falling into the pit.
If you expend your last movement action and have not met an enemy, you have a final opportunity to perform non-movement actions, after which your turn ends without proceeding to the Combat Phase.

When you encounter a monster, you enter the Combat Phase, dealt with below. Movement actions cannot be used after the Combat Phase, but you have a final opportunity to use any number of non-movement actions before your turn ends. If any of these actions result in monsters appearing (perhaps because you read a scroll of Create Foe), your turn ends immediately instead of moving to a second Combat Phase; you will fight the enemy on your next turn.

\subsection{Beginning Your Turn in a Room With a Monster} \label{beginRoomMonster}

If you happen to start your turn in the same room as a Monster, you skip the Movement Phase and move directly from the Beginning Phase to the Combat Phase. As an exception to this rule, if you fled from a Monster on your previous turn, and consequently are still in the room with it (see §\ref{combat}. Combat), you may spend a movement action moving away, then continue your turn as normal in the Movement Phase. If the movement action doesn’t result in your getting away (you tried to open a door and found it to be locked, for instance), you move to the Combat Phase anyway.

\section{Rooms}
A square on the board is \emph{unexplored} if it doesn’t cotain a Dungeon Tile. Initially all squares will be unexplored (except for the four containing the 2x2 Entry Tile). When you leave a room through an exit to an unexplored square, draw a tile from the Dungeon Tile Deck. Place it in the square with the small black arrow pointing away from the tile you just left. Consult the Tile Guide to determine what kind of dungeon features are present in the new tile. If there is a small line behind the black arrow, then the tile is a room. Draw a card from the Room Deck and follow its instructions. If the room contains a monster, draw from the appropriate level Monster Deck, and enter the Combat Phase (see §\ref{combat}. Combat).

When you enter an unexplored tile through an exit marked by a small white arrow, you draws a 2x2 Dungeon Tile and place it on the board, positioning it so that the small white arrow on the 2x2 tile points away from the tile you just left. If a 2x2 tile can’t fit on the board, either because it would cover an existing tile, or it would extend off the edge of the board, place a 1x2 Dungeon Tile instead. If the 1x2 tile can’t fit, place an ordinary 1x1 Dungeon Tile.  

\subsection{Doors}

If an exit contains a door, you can try to open it as a non-movement action. Draw a Door Card and follow its directions. Unless otherwise stated, the door opens and you may then pass through as a movement action. Put an open door token on the board to represent the open door. If the door is locked, put a locked door token on the board. Locked doors can only be opening by special items found in the dungeon, such as the Skeleton Key, and character abilities. If the door has a ongoing effect, such as the coin slot door that requires money whenever you pass, place the Door Card on the board using a standee.
When you pass through a door, place a 1x2 Dungeon Tile behind it. If a 1x2 Dungeon tile can’t fit, place a regular 1x1 Dungeon Tile.

As a movement action, you may search for \emph{secret doors} by selecting a solid wall and rolling 2d8. If the total is 12 or greater, you discover a secret door, and may pass through it as part of the same movement action that you discover it. Place a secret door token on the wall to represent it. Once discovered, anyone may use a secret door.
You cannot find a secret door in a wall leading to another part of the same tile.

\section{Spells} \label{spellcasting}
Only characters with the \emph{spellcasting} ability can cast spells. A character with this ability draws a number of Spell Cards equal to his Mind at the start of the game. Spells are cast in combat to help you defeat your enemies, and wear off when combat ends. Spells cannot be cast outside of combat (and generally aren’t useful anyway).
Most spells use the caster’s Mind value to determine the effect. When a spell is cast, discard its Spell Card. If you end your turn in town, you may draw one new Spell Card (see §\ref{town}. The Town).
When a monster uses the \textbf{cast spell} combat action, it draws the top card of the Spell Card, follows the instructions, then discards the card.

\section{Monsters} \label{monsters}

Monsters have the same five attributes as the characters, but behave randomly in battle in accordance with their Combat Chart. In the bottom half of the Frail Goblin card, you can see two columns. To determine the goblin’s action in battle, roll d8 and find the result in the left column. The goblin takes the action written in the right column on the same row. For example, suppose the player controlling the Frail Goblin rolls a 7. In this case, the cowardly goblin chooses to \textbf{flee}.
To the right on the monster’s image are symbols indicating the treasure you receive for slaying the monster. In the Frail Goblin’s case, you would receive 40 gold coins ( ) and an item ( ). Since the goblin is a level 1 monster, this item is drawn from the Level 1 Item Deck.
The middle text box gives general information about the monster. All monsters have an enemy type, which is one of \emph{Humanoid}, \emph{Animal}, \emph{Undead}, \emph{Thing}, \emph{Plant}, \emph{Dragon}, \emph{Demon}, and \emph{Qyzox}. All \emph{Undead} and \emph{Things} are immune to draining, poison, and status ailments. All \emph{Plants} are immune to poison and status ailments. Some individual monsters have their own resistances or vulnerabilities which are also listed here. In addition, if there are special rules governing the monster’s behavior not reflected in its Combat Chart, they are written here.
Some monsters can be taken as mounts once defeated. They are considered to be companions and confer the bonus described in the middle text box.

\section{Combat} \label{combat}
The Combat Phase begins when a player enters a room inhabited by one or more monsters. To determine combat order, each combatant rolls d8 and adds its Speed to the result. This total is known as that combatant’s initiative. Combatants act in descending order of initiative. If multiple combatants have the same initiative, they roll off among themselves to determine their order relative to each other. After every combatant has acted, the one with highest initiative acts again and battle continues in the same order until one side is victorious.
Any combatant that rolls a 1 on its initiative roll is \emph{surprised}. Surprised combatants act after all unsurprised combatants have acted, and have an effective Defence of 0 until their first action. If all of a combatant’s enemies are surprised, it can \textbf{flee} without needing to make a roll.

As a player, you have access to seven standard combat actions: \textbf{attack}, \textbf{flee}, \textbf{defend}, \textbf{cast spell}, \textbf{drink potion}, \textbf{throw potion}, and \textbf{read scroll}. When it is your action, you can choose one of these actions, or another action written on your Character Profile or Item Cards in /textbf{copperplate gothic}.
Monsters act according to their Combat Charts (see §\ref{monsters}, Monsters). Monsters use the actions \textbf{attack}, \textbf{flee}, \textbf{defend}, \textbf{cast spell}, \textbf{touch}, \textbf{teleport}, \textbf{mindstrike}, and sometimes perform other actions described in paragraphs on their Combat Charts.

To \textbf{attack}, roll d8 and add the result to your Attack. Subtract the enemy’s Defence from this number. The final total is the amount of damage done to the enemy. Subtract this total from the enemy’s current Health. If the enemy has 0 or less Health, it is defeated. If you possess multiple weapons, you must choose which weapon to use before you attack (and only add its bonus to your Attack value).
An attack roll of 1 is called a \emph{fumble}, and always misses. A roll of 8 is called a \emph{critical hit}. When you score a critical hit, roll another die and add it to your attack total (but don’t roll again if you get another 8).

When a combatant \textbf{defend}s, add 4 to its Defence until its next action.

To \textbf{flee}, roll d8 and add your Speed. The enemy with the highest Speed rolls d8 and adds its Speed. If your total is greater than your enemy’s, you escape successfully. You turn ends and you begin your next turn by moving away from the monsters (see §\ref{beginRoomMonster}. Beginning Your Turn in a Room with a Monster). When a monster flees, discard its Monster Card. If a combatant fails to flee, its Defence is reduced by 2 until its next action. If all enemies have Speed 0, you may automatically escape.

When a monster chooses \textbf{cast spell}, draw the top card of the Spell Deck. The monster casts that spell targetting the player. Players cast spells by discarding the corresponding Spell Card.

\textbf{read scroll} and \textbf{drink potion} work the same way as outside of combat - choose a potion or scroll,
follow the directions (flipping it over if it is face down), then discard it.

You can \textbf{throw} at \textbf{potion} at a monster. Roll d8 and add your Attack. Don’t add the bonus for your weapon. If the total exceeds the monster’s Defence, the potion shatters on the monster and it experiences the
effects of the potion.

When a monster tries to \textbf{touch} a combatant, roll d8 and add the result to its Attack. Subtract the target’s Defence from this number. If the total is positive, the target suffers the effect written on the monster’s Combat Chart after the word \textbf{touch}.

When a monster \textbf{teleport}s, it instantly leaves the battle. Discard its Monster Card; you don’t get any treasure.

A \textbf{mindstrike} is performed by rolling d8 and adding the result to one’s Mind. Subtract the target’s Mind from this value. Subtract the final total from the target’s Defence. Rolling a 1 or 8 doesn’t result in a fumble or critical hit.  

Monsters (and some player characters) can summon other monsters. When this happens, draw a card from the indicated Monster Deck. This card is the summoned monster. It performs the attack given on its Combat Chart across from the pentagram symbol ( ), then leaves - discard its card.
Any effect targeting a summoned monster instead targets the summoner. If the summoned monster casts a beneficial spell on itself, for instance, the combatant which summoned it would gain the effect. As another example, suppose Ron is wearing a Ring of Retribution, which reads: “When you take damage from an enemy’s \textbf{attack}, this ring deals 1 damage to that enemy.” A Young Qyzox summons a Giant Rat, which \textbf{attack}s Ron for 5 damage. Ron’s Ring of Retribution deals 1 damage to the Young Qyzox.

Sometimes a monster will cause the player to lose the rest of his turn. When this happens the player continues the battle on his next turn, as if that monster had just taken its turn. On that turn the player will start and end his turn in the combat phase, without going through a Beginning or Movement Phase. If second player enters the room before the player’s next turn, the monsters will ignore the second player.

When a monster is reduced to 0 or less Health, it is dead. When all enemies are dead, the player collects the treasure listed on their Monster Cards. Place the Monster Cards in the appropriate discard piles.

If the player runs away or otherwise leaves before all his enemies are defeated, then the enemies stay in the room and wait for somebody else to show up. In this case the monsters heal their injuries - next time a player enters the room the Monsters’ Health and other attributes will be at their starting values. Any Monsters that were killed before the player left the room are discarded.

\section{Status Ailments}
Players can experience temporary mental and physical infirmities known as \emph{status ailments}. Some status ailments only have an effect in combat and wear off as soon as combat ends. Others remain between turns and can only be removed in Town or by the appropriate items. Monsters can also experience status ailments, although \emph{undead}, \emph{things} and \emph{plants} are immune.

An \emph{afraid} combatant tries to \textbf{flee} at the next opportunity. This status is lost if it fails to escape, and also at end of combat.

An \emph{asleep} combatant doesn’t act and has an effective Defence of 0. Taking damage (or end of combat) causes
it to wake up.

A \emph{blind} combatant has its Attack and Defence both effectively reduced by 4. Blind players can’t read.

A \emph{confused} combatant acts randomly, choosing a random target for its actions. This means that it could potentially \textbf{attack} itself! Confused monsters act according to their Combat Chart, but choose their targets randomly.  Confused players roll d8 and consult the Confusion Special Rules Card. Confusion ends at end of combat.

If you are \emph{diseased}, you cannot heal Health.

\section{Attacks Types}
Some attacks have a special quality associated with them. These qualities are represented by an icon next to the attack. For example, the Dragon Hatching has an attack which reads

[ ] It breathes flame. Take d8 damage.

The fireball icon indicates that this is a \emph{fire} attack. If you are immune to fire, you take no damage from this attack.  If you are weak to fire, you take twice as much damage. Some traps have an icon next to their name. If you are immune to this attack type, the trap has no effect. If you weak to this attack type, you take twice as much damage from the trap.
The attack types are fire ( ), cold ( ), lightning ( ), acid ( ), poison ( ), draining ( ), and gaze ( ).

\section{Treasure}
When you find a chest, you can attempt to open it by drawing from the Chest deck. If the chest is locked and you
have no way to open locks, place the card in the room using a plastic stand, and leave it there for another player
to find. If you draw a trap card, the chest opens after you suffer the ill effects of the trap. If you get the chest
open, you can loot the contents, which are written along the bottom of the Chest Card. If you don’t decide to risk
opening the chest, put a chest token on the tile; the next player who enters the tile can open it by drawing a Chest
Card.
Item Cards represent things you find in the dungeon that can be carried. When you obtain one, place it in front of
you. When an effect says to lose an item, discard its Item Card. You can’t leave items lying around the dungeon,
but you can discard any unwanted items as a non-movement action. You can also trade items (and gold coins)
with other players.
Item Cards have a gold coin ( ) value in the lower right corner. This value does not count toward the 2000 total
players need to win the game. Players cannot sell their items. The gold coin value is present to indicate how much
the item costs to buy from a shop, and is relevant for some abilities (such as the Alchemist’s power to turns items
into gold).
Item Cards have an item type written on the bottom left of the card. The item types are Weapon, Body Armour, Shield,
Helmet, Boots, Accessory, Tool, and Treasure.
Items of type Body Armour, Shield, Helmet, and Boots must be equipped to be useful. Items can be equipped as a nonmovement action at any time outside of combat. Turn your unequipped items 90 degrees to the right to indicate
that they are not equipped. Once a battle begins, you cannot change your equipment, so choose carefully before
combat.
Items of type Weapon do not need to be equipped. Whenever you attack, you choose which weapon you are
going to attack with.
Items of type Tool can allow you to perform movement, non-movement, and combat actions, and sometimes have

passive abilities as well. It will be clear from the text of the tool when it can be used. If the tool has a sturdiness
value, then it may break after use. When you use such a tool, roll d8. If the result is greater than the tool’s
sturdiness value, it is broken and must be discarded.
Treasure items are the only ones that players can redeem for coins, which can be done in Town or at a Shop. They
generally aren’t otherwise useful.
If an item has the keyword Weight: X, your Speed ( ) is effectively reduced by X while it is in your possession.
If this brings your to 0 or below, you can’t perform movement actions until you drop it.
Cursed items beguile the owner into believing they hold incredible power. No character will purposefully discard
a cursed item, and if you draw one that can be equipped, you must have it equipped at all times. The curse can be
broken by ending your turn in Town.
When you receive a Potion ( ) or Scroll ( ), draw it from the Potion of Scroll Deck, but don’t look at it. Its nature
remains a mystery until identified, at which point you flip it over. If you’re too impatient to wait, you can drink an
unidentified Potion or read an unidentified Scroll as a non-movement action, or by spending your action in combat.
Flip it over and follow the directions. Potions and scrolls are discarded after use.

\section{The Town} \label{town}

Staircases up on level 1 lead the the Town. When you enter the Town, your turn ends, and you do any number of
the following:

\begin{itemize}
\item Restore your Health, Attack, Defence, Speed, and Mind to their starting values, if they
are lower.
\item Remove any persistent status ailments (i.e. \emph{blindness}, \emph{drunkeness}, and \emph{disease})
\item Discard any cursed items and Curse Cards.
\item Draw a Spell Card, if you have the spellcasting ability.

On your next turn, you re-enter the dungeon through the 2x2 Entry Tile.

\section{Dying} \label{dying}
When your Health ( ) reaches 0 or less you are dead. Discard all your Item Cards, Spell Cards, Potions, Scrolls,
your Character Profile, and any other cards in front of you. Any enemies you were fighting remain in room, but
discard any Monsters you killed before dying. If you want to keep playing, draw three new Character Profile
Cards and choose one. Next turn you enter the dungeon through the 2x2 Entry and begin a turn with your new
Character..

§16. Additional Rules
You may encounter other humans in the dungeon who wish to join you as companions. Companions don’t
participate directly in combat. Instead they boost your abilities in return for a fixed share of any loot you discover
When sharing gold with companions, round down the number of coins you receive. During the beginning phase
of your turn, you may dismiss a companion by discarding its card.
As an example of loot sharing, suppose you have four companions, three of whom take an “equal share of the

found,” and one who doesn’t take a fee. You kill a monster who has 350 . Giving an equal share to each of the
three companions who want a share, as well as an equal share for yourself, results in each of the four of you
receiving one fourth the total. One fourth of 300 is 87.5; thus you receive 87 for yourself.
Some monsters have actions that state “roll with the (enemy)” or “roll with the (enemy)”. In this case, both
you and the monster roll d8 and add the results to your Speed ( ) or Mind ( ) values. If your total is greater, then
you win the roll and the monster loses the roll, and vice versa.
Some cards ask you to roll Speed ( ) or Mind ( ) with target X. To do this, roll d8 and add your or value. If
the total is equal to or greater than the target, you succeed; otherwise you fail.
When a rule or card requires you to divide a quantity, round the result up unless instructed otherwise.

Fountain

Mysterious jets of water shoot from an ornate stone basin. If
you drink from the fountain, draw a Potion Card and place
it on this tile, then follow the directions on the card as if you
had just drunk the potion. The Potion Card remains on this
tile, and players on the tile can drink from the fountain once
each turn.

Springboard

As you enter this tile, a powerful mechanism causes the floor
to spring up beneath you, catapulting you through a one-way
door in the ceiling to the tile directly above you.
Flying characters may avoid this tile.
(Note: If the tile above you is a pit, it is possible to get stuck in a loop. If this
happens, choose one of the two tiles to end up on and continue your turn.)

Chriſtian Altar

These altars were built by a party of zealous missionaries before
they were devoured by the Dungeons’ denizens. If you end
your turn here, you may prostrate yourself before the altar in
fervent prayer. If you do, God will destroy your cursed items,
cure you of disease, and grant you a White Spell Card if you
can cast White Spells.

Chaſm

A gaping chasm leads down to dark depths! If you try to cross
the rickety bridge, roll against your speed. If successful, you
cross safely and may continue on the other side. If you fail,
move your piece to the square two floors directly below. (If
you are on level 3 or 4, the chasm is bottomless!).
Flying characters may avoid this tile.

Fan

A giant fan blows you down the hallway! Move your token
in the direction of the fan until you hit a wall or door, placing
new Dungeon Tiles in unexplored squares if necessary. Ignore
all features on the tiles you pass through; if you pass through a
new room place a Room Card face down on the tile.

Laboratory

You may identify any number of potions here (Turn over your
facedown Potion Cards).

Teleporter

As soon as you enter this tile, runes on the floor begin to glow,
and you find yourself somewhere new. Roll d6 and d8 and
move to the corresponding square on this floor.

To open a door, draw a Door Card and follow its
instructions. See §?, Doors.

Doors

Pit

You fall in a pit! Move your token to the same square on the floor
below. If that square is unexplored, place a new Dungeon Tile there
with the arrow oriented in the same direction as the tile you left.
Pits on level 4 are bottomless. If you fall through one, your character
is lost.
Flying characters avoid pits, and may spend a movement action to go
up or down a pit as if it were a staircase.

Springboard

As you enter this tile, a powerful mechanism causes the floor
to spring up beneath you, catapulting you through a one-way
door in the ceiling to the tile directly above you.
Flying characters may avoid this tile.
(Note: If the tile above you is a pit, it is possible to get stuck in a loop. If this
happens, choose one of the two tiles to end up on and continue your turn.)

Chriſtian Altar

These altars were built by a party of zealous missionaries before
they were devoured by the Dungeons’ denizens. If you end
your turn here, you may prostrate yourself before the altar in
fervent prayer. If you do, God will destroy your cursed items,
cure you of disease, and grant you a White Spell Card if you
can cast White Spells.

Chaſm

A gaping chasm leads down to dark depths! If you try to cross
the rickety bridge, roll against your speed. If successful, you
cross safely and may continue on the other side. If you fail,
move your piece to the square two floors directly below (If you
are on level 3 or 4, the chasm is bottomless!).
Flying characters may avoid this tile.

Name
Room
Corridor
Staircase

Draw a
Room
card?
Yes
No
Yes

Ice

No

Slide

No

Portal
chamber

Yes

Tunnel

N/A

Muddy
corridor
Chasm

No

Floodgates

Yes

Spikes

Yes

Giant fan

No

Pit

No

Locked Door
Dark
chamber
Laboratory
Library
Mysterious
Statues

N/A
No

Tome of
Knowledge
Crystal Ball

Yes

Ye Olde
Shoppe

No

Pentagram
Magical
Barrier

No
N/A

No

Yes
Yes
No

Yes

Description

When you go up or down a staircase, move your token to the same tile on
the level above or below. If the tile is unexplored put the corresponding
Staircase Tile there and draw a Room Card.
Leave through the door opposite the one through which you entered,
without using an action. Characters with Flying may avoid. Note: Sliding
on the ice requires momentum. You do not slide if, for example, you
teleport into an icy corridor or take of an Amulet of Flight while hovering
about some ice.
Move in the direction of the slide without using an action. Characters
with Flying may avoid.
You may enter the portal as an action, teleporting to a random location
on a random level. You may also escape from enemies into the portal by
successfully running away.
You may enter the tunnel as an action and come out at any other tunnel
mouth in the dungeon.
You get stuck in the mud and can not move again this turn, although you
can still perform other actions. Characters with Flying may avoid.
To cross the rickety bridge, roll against your speed. If you fail, you fall
to the square two levels below. Or you can just jump into the chasm (as
an action or when running away). Characters with Flying may avoid.
You may spend an action to pull the floodgate lever. If you do, the
current level and all levels below get washed away. Take all tiles off
these boards and move all players, monsters, items, and chests on any of
these levels to the same square on the level directly below. You may pull
the lever during combat by successfully running away.
If you entered the room due to a slide, ice, or giant fan, and the spikes
are on the opposing wall, lose 10 HP. If the spikes are on the floor and
you entered from above lose 10 HP.
Upon entering this tile, start moving in the direction opposite the fan,
and continue until you hit a wall. Don’t draw room cards for the tiles you
pass through. If you enter a fan room through the side containing the
fan, lose 10 HP.
You fall down to the level below. Characters with Climbing or Flying may
avoid pits, and move up through a pit from the level below.
Once a door is unlocked, it can be passed through by anyone.
You stumble around in the dark... you cannot read Scrolls in this room,
and upon leaving determine your exit randomly.
You may identify your Potions here.
You may identify your Scrolls here.
In this room are the statues of two monsters, each from a random
dungeon level. If you touch one, they both come to life and attack you. If
the statues are destroyed, new ones will appear upon re-entry.
If you end your turn here, you may draw a card from the Grey Spell Deck
(provided you can cast Black Spells).
If you end your turn here, you may draw a card from the Black Spell
Deck (provided you can cast Grey Spells), and restore your Mind to its
starting value.
When a shoppe appears on the board, it must be stocked from the
Treasure decks. Draw two cards from the DLV of the store, then two
cards from random DLVs. Also draw a Potion and a Scroll. If any wares
suit your fancy, you may purchase them. You may trade in items at half
value for store credit. Redraw worthless (0 GP) items. Any wares you
don’t buy remain in the shoppe to be purchased by other players; use
tokens to keep track of the cards.
Teleport to a random location on the current level.
When passing through the barrier, roll against Mind. If you fail, lose 10
HP.


